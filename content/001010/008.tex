% !TEX root = ../core/tetelfuzet.tex

\section{Definiálja a nemnegatív valós szám négyzetgyökét! Mivel egyenlő 
\texorpdfstring{$\sqrt{a^2}$}{sqrt(a**2)}?}
\label{008}

\begin{defin}[Négyzetgyök]
Legyen $a$ egy nemnegatív valós szám. Ekkor $\sqrt{a}$-n azt a nemnegatív valós
számot értjük, amelynek négyzete $a$. Azaz:
\[
\left.
\begin{array}{l}
  a \in \Reals \backslash \Reals^-\\
  \sqrt{a} \in \Reals \backslash \Reals^-
\end{array}
\right\} 
(\sqrt{a})^2 = a
\]

Megegyezés alapján:

$\sqrt{0} = 0$

$\sqrt{1} = 1$

Negatív valós számnak nincs valós négyzetgyöke.

Másképp mondva: a \dashuline{négyzetgyökvonás} olyan \emph{művelet}, amellyel a
szám négyzetéből meghatározzuk magát a számot. Így mindig két értéket ad:
$x^2 = a \Leftrightarrow x = \pm \sqrt{a}$
\end{defin}

Példák:
\[
\begin{array}{l|r}
  \sqrt{5^2} = 5   & \sqrt{(-5)^2} = |-5| = 5\\\hline
  \sqrt{x^2} = |x| & \\\hline
  \sqrt{x^4} = x^2 & \sqrt{x^6} = |x^3|\\\hline
  \sqrt{x^8} = x^4 & \sqrt{x^{10}} = |x^5|\\
\end{array}
\]

\textbf{Lásd még:}
