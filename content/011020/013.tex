\section{Mit értünk egy valós szám \texorpdfstring{$n$}{n}-edik gyökén 
  (\texorpdfstring{$n$}{n} pozitív egész)? Határozza meg 
  \texorpdfstring{$\sqrt[3]{27}; \sqrt[4]{256}; \sqrt[5]{-32}$}
  {27**(1/3); 256**(1/4); -32**(1/5)} értékét!}
\label{013}

\begin{defin}[$n$-edik gyök]
$a \in \Reals \backslash \Reals^-, n \in \Naturals: \sqrt[n]{a}$, mint
\dashuline{művelet értelmezése},
\begin{itemize}
  \item ha $n = 2k$ alakú (azaz páros), akkor $\sqrt[n]{a}$ olyan \emph{nem
    negatív valós} szám, amelynek $n$-edik hatványa $a$, azaz
    $n = 2k: \forall a \in \Reals \backslash \Reals^-: 
      (\sqrt[n]{a})^n \coloneqq a$
  \item ha $n = 2k+1$ alakú (azaz páratlan), akkor $\sqrt[n]{a}$ olyan 
    \emph{valós} szám, amelynek $n$-edik hatványa $a$, azaz
    $n = 2k + 1: \forall a \in \Reals: (\sqrt[n]{a})^n \coloneqq a$
\end{itemize}
\end{defin}

\[
  \sqrt[3]{27} = \sqrt[3]{3^3} = 3
\]
\[
  \sqrt[4]{256} = \sqrt[4]{4^4} = 4
\]
\[
  \sqrt[5]{-32} = \sqrt[5]{(-2)^5} = -2
\]

\textbf{Lásd még:}