% !TEX root = ../core/tetelfuzet.tex

\section{Mit értünk két vagy több egész szám legnagyobb közös osztóján?
\texorpdfstring{\\}{}Hogyan határozható meg?}
\label{001}

\begin{defin}[Közös osztó]
Két vagy több pozitív egész szám \dashuline{közös osztó}i azok a természetes
számok, amelyek mindegyik számnak osztói (a közös osztók száma véges).
\end{defin}

\begin{defin}[Legnagyobb közös osztó]
A közös osztók közül a legnagyobb nevezetes, ezt nevezzük \dashuline{legnagyobb
közös osztó}nak.

\uline{Jelölése:} $(a; b)$ vagy ha egyébként félreérthető, akkor $lnko(a; b)$,
angol szövegben $gcd(a; b)$ (\hunquote{greatest common divisor}) vagy $hcd(a;
b)$ (\hunquote{highest common factor}).
\end{defin}

\begin{method}[Legnagyobb közös osztó meghatározása]
A legnagyobb közös osztó meghatározásánál először felírjuk a számok
prímtényezős felbontását. Ezután a \emph{közös} prímszámok mindegyikénél
megkeressük a \emph{legkisebb} hatványkitevőjűt. Ezen legkisebb kitevőjű
prímszámhatványoknak a \emph{szorzata} lesz a számok legnagyobb közös osztója.
Ha nincs közös prímszám, akkor a legnagyobb közös osztó az egy.
\end{method}

\begin{note2}[Törtek egyszerűsítése]
Törteket a számláló és a nevező legnagyobb közös osztójával érdemes
egyszerűsíteni.
\end{note2}

\begin{method4}[Euklideszi-algoritmus]
Nagyon nagy számoknál a prímosztók meghatározása nehéz lehet, sok számolást
igényel. Két szám legnagyobb közös osztóját azonban meg tudjuk határozni a
prímszorzók megléte nélkül is a következő módon: írjuk le a két számot egymás
mellé. Vonjuk ki a nagyobból a kisebbiket, majd írjuk le az eredményt és a
kisebb számot egymás mellé. Az eljárást ismételjük addig, amíg a két szám nem
egyezik meg, a megegyező szám a két szám legnagyobb közös osztója.

Például: $lnko(1989; 867) = 51$, mert:
\[
\begin{array}{r|l}
  1989 & 867 \\
  1122 & 867 \\
  255 & 867 \\
  255 & 612 \\
  255 & 357 \\
  255 & 102 \\
  153 & 102 \\
  51 & 102 \\
  51 & 51 \\
\end{array}
\]
\end{method4}

\textbf{Lásd még:} \ref{002}, \ref{009}, \ref{010}
