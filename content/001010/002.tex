% !TEX root = ../core/tetelfuzet.tex

\section{Mit értünk két vagy több egész szám legkisebb közös többszörösén?
\texorpdfstring{\\}{}Hogyan határozható meg?}
\label{002}

\begin{defin}[Közös többszörös]
Két vagy több egész szám \dashuline{közös többszörös}e az a természetes szám,
amely mindegyik számnak többszöröse (közös többszörösből végtelen sok van).
\end{defin}

\begin{defin}[Legkisebb közös többszörös]
A közös többszörösök közül a legkisebb nevezetes, ezt az adott számok
\dashuline{legkisebb közös többszörösé}nek nevezzük.

\uline{Jelölése:} $[a; b]$ vagy ha egyébként félreérhető, akkor $lkkt(a; b)$,
angol szövegben $lcd(a; b)$ (\hunquote{lowest common denominator}) vagy $lcm(a;
b)$ (\hunquote{least common multiple}).
\end{defin}

\begin{method}[Legkisebb közös többszörös meghatározása]
A legkisebb közös többszörös meghatározásánál a számokat először prímtényezők
szorzatára bontjuk. Ezután a bennük szereplő \emph{összes} prímtényezőt az
előforduló \emph{legnagyobb} hatványkitevővel vesszük és ezeket
\emph{összeszorozzuk}.
\end{method}

\begin{note2}[Törtek közös nevezőre hozása]
Törtek közös nevezőre hozásánál célszerű a nevezők legkisebb közös többszörösét
venni a közös nevezőnek.
\end{note2}

\begin{method2}
Ha nem szeretnénk/nem tudjuk meghatározni a számok prímtényezős felbontását,
akkor a számelmélet alaptételéből következően a legkisebb közös többszöröst
számolhatjuk az $lkkt(a; b) = \dfrac{a \cdot b}{lnko(a; b)}$ képlettel.
(Természetesen ez \hunquote{fordítva} is működik, azaz $lnko(a; b) = \dfrac{a
\cdot b}{lkkt(a; b)}$)
\end{method2}

\textbf{Lásd még:} \ref{001}
