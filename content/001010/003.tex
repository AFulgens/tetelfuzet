% !TEX root = ../core/tetelfuzet.tex

\section{Milyen számot nevezünk prímszámnak?
\texorpdfstring{\\}{}Mikor mondjuk, hogy két vagy több egész szám relatív
prím?}
\label{003}

\begin{defin}[Valódi osztó]
Egy szám egyen és önmagán kívüli osztóit a szám \dashuline{valódi osztó}inak
nevezzük (tehát az egy és a szám nem valódi osztó).

Pl.: $6$ osztói: $1; 2; 3; 6$, melyekből a $2$ és a $3$ valódi osztók.
\end{defin}

\begin{defin}[Prímszám]
Azokat az egynél nagyobb természetes számokat, amelyeknek pontosan két osztójuk
van (az egy és önmaga) \dashuline{prímszám}oknak, vagy törzsszámoknak nevezzük.

\end{defin}

\begin{note2}[A $100$-nál kisebb prímszámok] $2$, $3$, $5$, $7$, $11$, $13$,
$17$, $19$, $23$, $29$, $31$, $37$, $41$, $43$, $47$, $53$, $59$, $61$, $67$,
$71$, $73$, $79$, $83$, $89$, $97$
\end{note2}

\begin{defin2}[Ikerprímek]
Azokat a prímszámokból álló párokat, amelyek különbsége 2,
\dashuline{ikerprímek}nek nevezzük. Például a $71, 73$ egy ikerprím pár.
\end{defin2}

\begin{theorem}[Prímek számossága]
Végtelen sok prímszám van.
\end{theorem}

\begin{proof3}
\mbox{ }(Eukleidész: Elemek, 9. könyv, 20. tétel, i. e. 300 körül) Legyen $S$
prímek egy véges halmaza. Vegyük a következő számot:
\[
N = 1 + \prod_{p \in S}p
\]
Mivel $N$ természetes szám, ezért legalább egy prímszámmal osztható, ha
prímszám, akkor saját magával. $N$ egyik osztója sem lehet benne az $S$
halmazban, mert az azokkal való osztás során $1$ maradék képződik. Emiatt $N$
prím osztóival tudjuk bővíteni az $S$ halmazt. Ebből következően minden véges,
prímszámokból álló halmaz tetszőleges sokszor bővíthető.
\end{proof3}

\begin{defin}[Összetett szám]
Azokat az egynél nagyobb természetes számokat, amelyeknek kettőnél több
osztójuk van \dashuline{összetett szám}oknak nevezzük.
\end{defin}

\begin{corollary}
A definíciókból következően az egyet sem prímszámnak, sem összetett számnak nem
tekintjük.
\end{corollary}

\begin{defin}[Relatív prímek]
Ha két pozitív egész szám legnagyobb közös osztója egy, akkor azokat
\dashuline{relatív prímszámok}nak nevezzük.

Például $lnko(25; 36) = 1$, tehát a $25$ és a $36$ relatív prímek.
\end{defin}

\textbf{Lásd még:} \ref{001}
