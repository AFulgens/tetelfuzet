% !TEX root = ../../core/tetelfuzet.tex

\section{Mit nevezünk egy valós szám normálalakjának? Írja fel a következő 
számok normálalakját:
\texorpdfstring{$\quad0,000173;$}{ 0,000173;}
\texorpdfstring{$\quad 58200000;$}{ 58200000;}
\texorpdfstring{$\quad\dfrac{78}{582}$}{ 78/582}
!}

\begin{defin}[Pozitív valós szám normálalakja]
Olyan \emph{kéttényezős} szorzat, amelynek:
\begin{itemize}
  \item egyik tényezője $1$ vagy $1$-nél nagyobb, de $10$-nél kisebb valós 
    szám,
  \item másik tényezője $10$-nek egész kitevőjű hatványa.
\end{itemize}
\end{defin}

\begin{defin}[Negatív valós szám normálalakja]
Olyan \emph{kéttényezős} szorzat, amelynek:
\begin{itemize}
  \item egyik tényezője $-1$ vagy $-1$-nél kisebb, de $-10$-nél nagyobb valós
    szám;
  \item másik tényezője $10$-nek egész kitevőjű hatványa.
\end{itemize}
\end{defin}

\begin{note}
$0$ normálalakja megegyezés szerint $0$.
\end{note}

\[
  0,000173 = 1,73 \cdot 10^{-4}
\]

\[
  58200000 = 5,82 \cdot 10^7
\]

\[
  \frac{78}{582} = 0,1340 = 1,34 \cdot 10^{-1}
\]

\textbf{Lásd még:} 
