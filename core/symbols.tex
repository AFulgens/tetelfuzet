% !TEX root = ../core/tetelfuzet.tex

\section*{Jelölésjegyzék}

A jelöletlen kérdések, definíciók, tételek, stb. a középiskolai törzsanyag
részét képezik. A középszintű érettségin tudásuk szükségeltetik.

A $\dagger$-rel (tőrrel) jelölt kérdések, definíciók, tételek, stb. a
középiskolai törzsanyag ismeretével megérthetőek, megértésük lehet, hogy
nagyobb odafigyelést, mélyebb végiggondolást igényel. A középszintű érettségin
tudásuk a jobb jegyek eléréséhez szükséges lehet.

A $\maltese$-tel (kereszttel) jelölt kérdések, definíciók, tételek, stb. a
középiskolai emelt szint részét képezik. Az emelt szintű érettségin tudásuk
szükségeltetik.

A $\pitchfork$-val (villával) jelölt kérdések, definíciók, tételek, stb. a
középiskolai emelt szintű anyagok ismeretével megérthetőek, de lehet, hogy
szükséges hozzájuk egyetemi szintű tananyag ismerete. A bizonyítást lehet, hogy
mellőzzük, egyetemi szintű tanagyag általában szükséges lenne hozzá.

A $\bigstar$-gal (csillaggal) jelölt kérdések, definíciók, tételek, stb. nem
részei a középiskolai anyagnak, megértésükhöz valószínű, hogy szükség van
egyetemi szintű matematikai ismeretekre. Bizonyos esetekben az ezzel jelölt
bizonyításoktól jelen kiadványban, bonyolultságuk okán eltekintünk.

\begin{figure}[h!]
\[
  \begin{array}{|c|c|c||c|c|c|}\hline
  \multicolumn{2}{|c|}{\text{Betű}} & \text{Név} 
  & \multicolumn{2}{|c|}{\text{Betű}} & \text{Név}
    \\\hline\hline
  \text{A} & \alpha & \text{alfa}
  & \text{N} & \nu & \text{nu}
    \\\hline
  \text{B} & \beta & \text{béta}
  & \Xi & \xi & \text{kszí}
    \\\hline
  \Gamma & \gamma & \text{gamma}
  & \text{O} & \text{o} & \text{omikron}
    \\\hline
  \Delta & \delta & \text{delta}
  & \Pi & \pi \text{ (vagy }\varpi \text{)}& \text{pí}
    \\\hline
  \text{E} & \varepsilon \text{ (vagy }\epsilon \text{)} &
    \text{epszilon}
  & \text{P} & \varrho \text{ (vagy }\rho \text{)}& \text{ró}
    \\\hline
  \text{Z} & \zeta & \text{dzéta}
  & \Sigma & \sigma \text{ (vagy }\varsigma \text{)}& \text{szigma}
    \\\hline
  \text{H} & \eta & \text{éta}
  & \text{T} & \tau & \text{tau}
    \\\hline
  \Theta & \theta \text{ (vagy }\vartheta\text{)} & \text{théta}
  & \Upsilon & \upsilon & \text{üpszilon}
    \\\hline
  \text{I} & \iota & \text{ióta}
  & \Phi & \varphi \text{ (vagy }\phi \text{)}& \text{fí}
    \\\hline
  \text{K} & \kappa \text{ (vagy }\varkappa\text{)} & \text{kappa}
  &\text{X} & \chi & \text{khí}
    \\\hline
  \Lambda & \lambda & \text{lambda}
  & \Psi & \psi & \text{pszí}
    \\\hline
  \text{M} & \mu & \text{mu}
  & \Omega & \omega & \text{ómega}
    \\\hline
  \end{array}
\]
\caption{A görög ábécé betűi}
\end{figure}

Az egyes tételek bizonyításai a félreértések elkerülése végett indentálásra
kerültek és a bal oldalon folytonos függőleges vonallal jelölésre kerültek.

Egy bizonyítás végét a \hunquote{\qed} (\hunquote{quod erat demonstrandum}, am.
\hunquote{ezt kellett bizonyítanunk}) betűkapcsolat jelzi.

Az egyes definíciókban az éppen definiált kifejezés \dashuline{szaggatott
aláhúzással} került kiemelésre, míg a fontos részletek \emph{kurzívval} lettek
szedve.

$\sum$ a szummázás jele, azaz
  $\sum\limits_{i=0}^{n}a_i = a_1 + a_2 + \dotsb + a_{n-1} + a_n$

$\prod$ a produktum jele, azaz
  $\prod\limits_{i=0}^{n}a_i = a_1 \cdot a_2 \dotsm a_{n-1} \cdot a_n$

$\forall$ a \hunquote{minden}, $\exists$ a \hunquote{létezik} jele. Azaz
$\forall a$ jelentése \hunquote{minden $a$-ra}, míg $\exists a$ jelentése
\hunquote{létezik olyan $a$, amire}. A $\exists!$ pedig az \hunquote{egyértelmű
létezés} jele, azaz $\exists! a$ azt jelenti, hogy \hunquote{pontosan egy olyan
$a$ létezik, amire}.

Néhányszor előfordul, hogy egy tétel gyengébb minősítést kap, mint a tétel
bizonyítása. Vagy éppen a bizonyítás maga gyengébb minősítésű, mint azon
lemmák/tételek bizonyítása, amelyekből a bizonyítás építkezik. Ez megszokott,
ne lepődjünk meg rajta.

\blitza-mal (villámmal) jelöljük az ellentmondást, amelyet indirekt
bizonyítások végén szándékozunk elérni.

Gyakran pongyolán fogalmazunk, ha geometriában szakaszok hosszáról van szó, és
azok közötti összefüggésekről. Írjuk azt, hogy \hunquote{szakaszok szorzata},
amikor szakaszok hosszának szorzatáról beszélünk és írjuk azt, hogy $AB$,
amikor az $\overline{AB}$ szakaszra gondolunk, vagy ha épp képletben annak
hosszára, azaz $|\overline{AB}|$-re.

Az intervallumok végeinek jelei $[$; $($; $]$ és $)$, a gömbölyű zárójel azt
jelzi, hogy az intervallum azon vége nyílt, míg a szögletes zárójel azt, hogy
az intervallum azon vége zárt. Azaz $[a; b)$ esetén az intervallum $a$-tól
(beleértve) $b$-ig (bele nem értve) tart (szokás még a nyílt véget
\hunquote{kifelé fordított} szögletes zárójellel jelölni, ettől a jelöléstől
tipográfiai okok miatt eltekintünk).

$\coloneqq$ jelentése \hunquote{legyen egyenlő}, szokás még \hunquote{definiáló
egyenlőség}-nek is hívni.
\cleardoublepage
