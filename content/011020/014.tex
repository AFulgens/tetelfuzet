% !TEX root = ../../core/tetelfuzet.tex

\section{Igazolja a következő azonosságokat! Milyen kikötéseket kell tenni 
  \texorpdfstring{$a$}{a}-ra, 
  \texorpdfstring{$b$}{b}-re, 
  \texorpdfstring{$n$}{n}-re és 
  \texorpdfstring{$k$}{k}-ra?}
\label{014}

\subsection{
\texorpdfstring
  {$\sqrt[n]{ab} = \sqrt[n]{a} \cdot \sqrt[n]{b}$}
  {(ab)**(1/n) = a**(1/n) * b**(1/n)}
}

Informálisan:

$\rightarrow$ Szorzat $n$-edik gyöke egyenlő a tényezők $n$-edik gyökének
szorzatával.

$\leftarrow$ $n$-edik gyökös kifejezések szorzata egyenlő a gyök alatti
kifejezések szorzatának $n$-edik gyökével.

\begin{proof}

$ \sqrt[n]{ab} 
 \xlongequal{\ref{def:nthroot}.}
  (ab)^{\frac{1}{n}}
 \xlongequal{\ref{q:powereq}/\ref{theo:powermul}}
  a^{\frac{1}{n}} \cdot b^{\frac{1}{n}} 
 \xlongequal{\ref{def:nthroot}.} 
  \sqrt[n]{a} \cdot \sqrt[n]{b}
$

Feltéve, hogy: $n \in \Naturals, n >1$ és
\begin{itemize}
  \item ha $n$ páros, akkor $a, b \in \Reals \backslash \Reals^-$;
  \item ha $n$ páratlan, akkor $a, b \in \Reals$.
\end{itemize}
\end{proof}

\subsection{
\texorpdfstring
  {$\sqrt[n]{\frac{a}{b}} = \frac{\sqrt[n]{a}}{\sqrt[n]{b}}$}
  {(a/b)**n = a**n / b**n}
}

Informálisan:

$\rightarrow$ Tört $n$-edik gyöke egyenlő a számláló $n$-edik gyökének és a
nevező $n$-edik gyökének hányadosával.

$\leftarrow$ $n$-edik gyökös kifejezések hányadosa egyenlő a gyök alatti
kifejezések hányadosának $n$-edik gyökével.

\begin{proof}

$ \sqrt[n]{\Big(\dfrac{a}{b}\Big)} 
 \xlongequal{\ref{def:nthroot}.}
  \Big(\dfrac{a}{b}\Big)^{\frac{1}{n}}
 \xlongequal{\ref{q:powereq}/\ref{theo:powerdiv}}
  \dfrac{a^{\frac{1}{n}}}{b^{\frac{1}{n}}}
 \xlongequal{\ref{def:nthroot}.}
  \dfrac{\sqrt[n]{a}}{\sqrt[n]{b}}
$

Feltéve, hogy: $n \in \Naturals, n >1$ és
\begin{itemize}
  \item ha $n$ páros, akkor $a \in \Reals \backslash \Reals^-, b \in \Reals^+$;
  \item ha $n$ páratlan, akkor $a, b \in \Reals, b \neq 0$.
\end{itemize}
\end{proof}
	
\subsection{
\texorpdfstring
  {$(\sqrt[k]{a})^n = \sqrt[k]{a^n}$}
  {(a**(1/k))**n = (a**n)**(1/k)}
}
Informálisan:

Hatvány $k$-adik gyöke egyenlő az alap $k$-adik gyökének hatványával.

\begin{proof}

$ (\sqrt[k]{a})^n 
 \xlongequal{\ref{def:nthroot}.}
  (a^{\frac{1}{k}})^n
 \xlongequal{\ref{q:powereq}/\ref{theo:powerpow}}
  a^{\frac{n}{k}}
 \xlongequal{\ref{def:nthroot}.}
  \sqrt[k]{a^n}
$

Feltéve, hogy: $k \in \Naturals, k >1, n \in \Integers$ és
\begin{itemize}
  \item ha $n$ páros, akkor $a \in \Reals \backslash \Reals^-$;
  \item ha $n$ páratlan, akkor $a \in \Reals$.
\end{itemize}
\end{proof}
	
\textbf{Lásd még:} 
