% !TEX root = ../../core/tetelfuzet.tex

\section{Hogyan definiáljuk az \texorpdfstring{$a$}{a} valós szám pozitív egész
kitevőjű hatványát?}
\label{006}

\begin{defin}[Hatvány]
\label{def:power}
Legyen $a \in \Reals$, $n \in \Naturals \backslash \{1\}$, ekkor $a^n$ olyan
\dashuline{$n$ tényezős szorzat}ot jelent, amelynek minden tényezője $a$, azaz:
\[
a^n = \underbrace{a \cdot a \cdot a \cdot a \cdots{ } a}_{n \text{ db}}
\]
Elnevezések:

$a$: hatványalap

$n$: hatványkitevő

$a^n$: hatványérték vagy hatvány

Megállapodás alapján $a^1 = a$ (ezek alapján már $n \in \Naturals$).
\end{defin}

\textbf{Lásd még:} \ref{007}, \ref{008}, \ref{009}, \ref{012}
