% !TEX root = ../../core/tetelfuzet.tex

\section{Mit jelent \texorpdfstring{$\log_{a}b$}{log\textunderscore{}a(b)}? 
Milyen kikötéseket kell tenni \texorpdfstring{$a$}{a}-ra és
\texorpdfstring{$b$}{b}-re?}
\label{016}

\begin{defin}[Logaritmus]
\label{def:log}
$a$ alapú logaritmus $b$ jelenti \dashuline{azt a kitevő}t, amelyre $a$-t
hatványozva $b$-t kapunk. Azaz:
\[
  a^{\log_{a}b} = b
\]
ahol $a, b \in \Reals^+, a \neq 1$.

Elnevezések:

$a$: a logaritmus alapja

$b$: a logaritmus numerusza
\end{defin}

Megállapodás alapján:
\[
  \begin{array}{lr}
    \log_{a}a = 1 & \log_{a}1 = 0
  \end{array}
\]

Tehát definíció szerint:
\[
  \begin{array}{lr}
    a^{\log_{a}a} = a & a^{\log_{a}1} = 1\\
  \end{array}
\]

\textbf{Lásd még:} \ref{009}, \ref{012}
