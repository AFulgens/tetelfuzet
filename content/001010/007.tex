% !TEX root = ../core/tetelfuzet.tex

\section{Igazolja a következő azonosságokat (\texorpdfstring{$a, b$}{a, b}
valós számok, \texorpdfstring{$n, k$}{n, k} pozitív egészek)!}
\subsection{\texorpdfstring{$(ab)^n = a^n \cdot b^n$}{(ab)**n = a**n * b**n}}
Informálisan:

$\rightarrow$ Szorzat $n$-edik hatványa egyenlő a tényezők $n$-edik hatványának
szorzatával.

$\leftarrow$ Azonos kitevőjű hatványokat úgy is szorozhatunk, hogy az alapok
szorzatát a közös kitevőre emeljük.

\begin{proof}

$(ab)^n
 \xlongequal{\ref{def:power}.}
 \underbrace{(ab)(ab)(ab) \dotsm (ab)}_{n \text{ db}}
 \xlongequal{\ref{def:commutat}., \ref{def:assoc}.}
 \underbrace{a \cdot a \cdot a \dotsm a}_{n \text{ db}} \cdot
   \underbrace{b \cdot b \cdot b \dotsm b}_{n \text{ db}}
 \xlongequal{\ref{def:power}.}
 a^n \cdot b^n
$
\end{proof}

\begin{corollary2}
$(a \cdot b \cdot c \cdot e \cdot f \dotsm k)^n 
 =
 a^n \cdot b^n \cdot c^n \cdot e^n \cdot f^n \dotsm k^n
$
\end{corollary2}

\subsection{
\texorpdfstring{$\Big(\dfrac{a}{b}\Big)^n = \dfrac{a^n}{b^n} \quad (b \neq 0)$}
{(a/b)**n = a**n / b**n (b =/= 0)}}
Informálisan:

$\rightarrow$ Tört $n$-edik hatványa egyenlő a számláló és a nevező $n$-edik 
hatványának a hányadosával.

$\leftarrow$ Azonos kitevőjű hatványokat úgy is oszthatunk, hogy az alapok 
hányadosát a közös kitevőre emeljük.

\begin{proof}

$\Big(\dfrac{a}{b}\Big)^n 
 \xlongequal{\ref{def:power}}
 \underbrace{\frac{a}{b} \cdot \frac{a}{b} \cdot \frac{a}{b}
   \dotsm \frac{a}{b}}_{n \text{ db}}
 \xlongequal{\ref{def:commutat}., \ref{def:assoc}.}
 \overbrace{\underbrace{\frac{a \cdot a \cdot a \dotsm a}
   {b \cdot b \cdot b \dotsm b}}_{n \text{ db}}}^{n \text{ db}}
 \xlongequal{\ref{def:power}.}
 \dfrac{a^n}{b^n}
$
\end{proof}

\subsection{\texorpdfstring{$(a^n)^k = a^{n \cdot k}$}{(a**n)**k = a**(n * k)}}
Informálisan:

Hatványt úgy is hatványozhatunk, hogy az alapot a kitevők szorzatára emeljük.

\begin{proof}

$(a^n)^k
 \xlongequal{\ref{def:power}.}
 \underbrace{a^n \cdot a^n \dotsm a^n}_{k \text{ db}} 
 \xlongequal{\ref{def:power}.}
 \underbrace{\underbrace{a \cdot a \cdot a \dotsm a}_{n \text{ db}}
   \cdot \underbrace{a \cdot a \cdot a \dotsm a}_{n \text{ db}} \dotsm
   \underbrace{a \cdot a \cdot a \dotsm a}_{n \text{ db}}}_{k \text{ db}}
 \xlongequal{\ref{def:power}.} a^{n \cdot k}
$
\end{proof}

\textbf{Lásd még:}
