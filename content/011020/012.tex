% !TEX root = ../../core/tetelfuzet.tex

\section{Hogyan definiálja egy pozitív szám \texorpdfstring{$0$}{0}, negatív
  egész és racionális kitevőjű hatványát?}
\label{012}

A hatványfogalom kiterjesztése úgy történik, hogy a már megismert
hatványazonosságok érvényben maradjanak.

\begin{defin}[Nulla kitevőjű hatvány értelmezése]
Bármely nullától különböző valós szám nulladik hatványa $1$. Azaz:
\[
  a \in \Reals \backslash \{0\}: a^0 = 1
\]
\end{defin}

\begin{defin}[Negatív egész kitevőjű hatvány értelmezése]
Bármely nullától különböző valós szám negatív egész kitevőjű hatványa egyenlő
az alap ellentett kitevőjű hatványának reciprokával. Azaz:
\[
  a \in \Reals \backslash \{0\}, n \in \Naturals: a^{-n} = \dfrac{1}{a^n}
\]
\end{defin}

\begin{defin}[Racionális kitevőjű hatvány értelmezése]
Egy pozitív valós $a$ szám $\dfrac{m}{n}$-edik hatványa a pozitív $a$ alapnak
az $m$-edik hatványából vont $n$-edik gyöke. Azaz:
\[
  a \in \Reals^+, m, n \in \Integers, n > 1: a^{\frac{m}{n}} = \sqrt[n]{a^m}
\]

Vagyis a tört kitevőjű hatvány gyökös alakra írható át és fordítva, a gyökös
alak tört kitevőjű hatvány alakra írható.
\end{defin}

\begin{defin4}[Irracionális kitevőjű hatvány értelmezése]
Egy pozitív $a$ valós szám irracionális kitevőjű hatványát két ekvivalens módon
vezethetjük be ($a \in \Reals^+, b \in \Irrationals$):
\begin{enumerate}[i)]
\item az ún. közelítő módszerrel: $a^b = \lim\limits_{x \to b} a^x$
\item logaritmus és az $e$ szám segítségével: $a^b = e^{b \cdot \ln a}$
\end{enumerate}
\end{defin4}

\textbf{Lásd még:} \ref{006}, \ref{007}, \ref{008}, \ref{009}, \ref{013}, \ref{014}, \ref{016}, \ref{017}
