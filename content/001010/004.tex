% !TEX root = ../core/tetelfuzet.tex

\section{Mit jelent az, hogy a valós számokra értelmezett összeadás és szorzás
kommutatív, asszociatív, illetve a szorzás az összeadásra nézve disztributív?}
\label{004}

Az alábbi definíciókban legyenek $a, b, c \in \Reals$.

\begin{defin}[Kommutativitás]
\label{def:commutat}
am. \hunquote{felcserélhetőség}.

Összeadáskor a tagok sorrendjét felcserélhetjük, az összeg értéke nem változik
meg, azaz: $a + b = b + a$

Szorzáskor a tényezők sorrendjét felcserélhetjük, a szorzat értéke nem változik
meg, azaz: $a \cdot b = b \cdot a$
\end{defin}

\begin{defin}[Asszociativitás]
\label{def:assoc}
am. \hunquote{csoportosíthatóság}.

Három vagy annál több tag összeadásakor a tagokat tetszés szerint
csoportosíthatjuk, az összeg értéke nem változik meg, azaz:
$(a + b) + c = a + (b + c)$

Három vagy annál több tényező szorzásakor a tényezőket tetszés szerint
csoportosíthatjuk, a szorzat értéke nem változik meg, azaz:
$(a \cdot b) \cdot c = a \cdot (b \cdot c)$
\end{defin}

\begin{defin}[Disztributivitás]
am. \hunquote{széttagolhatóság}.

Ha a szorzandó több tagú, akkor a szorzandó minden tagját külön-külön
megszorozzuk a szorzóval és az így kapott értékeket összeadjuk, vagy előbb
összeadjuk a tagokat és ezután szorzunk, az eredmény ugyanaz lesz. Azaz:
$(a + b) \cdot c = a \cdot c + b \cdot c$

Fordított alkalmazását kiemelésnek hívjuk.
\end{defin}

\textbf{Lásd még:}
