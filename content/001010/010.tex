% !TEX root = ../core/tetelfuzet.tex

\section{Mi a számelmélet alaptétele?}
\label{010}

\begin{theorem}[A számelmélet alaptétele]
Bármely összetett szám a tényezők sorrendjétől eltekintve egyértelműen
felírható prímszámok szorzataként.
\end{theorem}

\begin{proof4long}
\begin{lemma4}[Bézout lemmája]
\label{lemma:bezout}
Legyen $a, b \in \Integers \backslash \{0\}$. Legyen $d = lnko(a; b)$. Ekkor:
\[
  \exists x, y \in \Integers: ax + by = d
\]
Továbbá:
\begin{enumerate}[i)]
\item $d$ a legkisebb $ax + by$ alakban felírható egész szám
\item minden $ax + by$ alakú egész $d$ többszöröse
\end{enumerate}
Az $x, y$ számokat az $(a; b)$ pár Bézout együtthatóinak hívjuk, és nem
egyértelműek (adott számpárra végtelen sok Bézout együtthatópár létezik).

Pl.: $a \coloneqq 12, b \coloneqq 42 \Rightarrow d = lnko(12; 42) = 6$. Ekkor
néhány lehetséges Bézout együtthatópár:
$(-10; 3); (-3; 1); (4; -1); (11; -3); (18; -5);$ stb.
\end{lemma4}\\[-1em]
\begin{proof4}[35.1pt]
Tudjuk, hogy egy $b$ nem nulla egész szám bármely egész számmal vett osztás
utána maradéka szigorúan kisebb, mint $|b|$.

Adott nem nulla $a$ és $b$ egészekre létezik nem nulla egész
$d = as + bt$ ($s, t \in \Integers$), aminek abszolútértéke minimális az
$ax + by$ alakú egészek között. Feltehetjük, hogy $d > 0$ (mert ha nem, akkor
$s, t$ előjelét annak megfelelően változtassuk meg).

Ekkor ha $a$-t vagy $b$-t osztjuk $d$-vel, akkor a maradék is $ax + by$ alakú
lesz, hiszen $d = as + bt$ valahányszorosát vontuk ki a számból. Másrészt a
maradék szigorúan kisebb, mint $d$. Ezek miatt a maradék biztosan $0$, tehát 
$d$ osztója $a$-nak és $b$-nek is.

Ha létezik $c$ másik közös osztója $a$-nak és $b$-nek, akkor $c$ szintén osztja
$as + bt = d$-t, ami azt jelenti, hogy $d$ az $a$ és $b$ számok legnagyobb
közös osztója.
\end{proof4}\\[-1em]
\begin{lemma3}[Eukleidész lemmája]
\label{lemma:euclid}
Legyen $p$ prímszám, ami osztja $a \cdot b$-t, ahol $a, b \in \Integers$. Ekkor
$p$ osztója $a$-nak vagy $b$-nek (esetleg mindkettőnek).

Általánosabban: $n | ab, lnko(n; a) = 1 \Rightarrow n|b$.
\end{lemma3}\\[-1em]
\begin{proof4}[35.1pt]
Legyenek $x, y \in \Integers, lnko(x; y) = 1$. Ekkor
$\exists r, s \in \Integers$: $rx + sy = 1$.

Legyenek továbbá $a, b, n \in \Integers, lnko(a; b) = 1, n | ab$.
\ref{lemma:bezout}. miatt $\exists r, s: rn + sa = 1$.

Szorozva mindkét oldalt $b$-vel azt kapjuk, hogy: $rnb + sab = b$.

$rnb$ osztható $n$-nel, $sab$ osztható $ab$-vel, ami osztható $n$-nel. Emiatt
összegük, ami $b$, szintén osztható lesz $n$-nel.
\end{proof4}\\
\textbf{Létezik:} Teljes indukcióval. Könnyen látható, hogy
\dashuline{$n = 2, 3, 4$ esetén} prímszámok szorzatáról beszélünk. Tegyük fel,
hogy $n-1$-nek létezik prímtényezős felbontása. \dashuline{Tekintsük $n$-t}:
\begin{itemize}
\item $n$ prím: kész vagyunk $\checkmark$
\item $n$ nem prím: $\exists a, b \in \Integers: 1 < a \leq b < n = ab$. Az
indukciós feltétel miatt $a = p_1p_2\dotsm p_n$ és $b = q_1q_2\dotsm q_m$ (ahol
$i \in [1 .. n]: p_i$ és $j \in [1 ..m]: q_j$ mind prímek), tehát
$ab = p_1p_2\dotsm p_nq_1q_2\dotsm q_m$ is prímek szorzata $\checkmark$
\end{itemize}\\
\textbf{Egyértelmű:} Indirekt módon. Tegyük fel, hogy
$s = p_1p_2\dotsm p_n = q_1q_2\dotsm q_m$, ahol
$i \in [1 .. n]: p_i, j \in [1 .. m]: q_j$ mind prímek és legalább egy
különböző elem van a két szorzatban. Be kell látnunk, hogy $m = n$ és $q_j$ a
$p_i$-k megfelelő sorrendben rendezésével egyenlőek.

\ref{lemma:euclid}. miatt $p_1$ valamelyik $q_j$-nek osztója. Tegyük fel, hogy
$p_1$ osztja $q_1$-et (ha nem, akkor indexeljük újra ennek megfelelően a
$q$-kat). Mivel $q_1$ prím, ezért csak $1$ és önmaga az osztói, tehát
$p_1 = q_1$, azaz: $\dfrac{s}{p_1} = p_2\dotsm p_n = q_2\dotsm q_m$. Ezt a
gondolatot alkalmazhatjuk $n$-szer, tehát tudjuk, hogy $n \leq m$. Ekkor a
maradékunk: $\dfrac{s}{p_1p_2\dotsm p_n} = 1 = q_{n+1}\dotsm q_m$. Ez
lehetetlen, mivel prímek szorzata nem lehet $1$ \blitza $\Rightarrow m = n$ és
minden $q_j$ megegyezik valamely $p_i$-vel.

\end{proof4long}

\begin{defin2}[Kanonikus alak]
Egy szám \dashuline{kanonikus alakjá}nak azt a prímtényezős felbontását
nevezzük, amiben a tényezők növekvő sorrendbe vannak rendezve és a megegyező
prímeket hatványként írjuk le, például $79380$ kanonikus alakja:
$2^2 \cdot 3^4 \cdot 5 \cdot 7^2$.
\end{defin2}

\begin{method2}[Kanonikus alak meghatározása]
Egy szám kanonikus alakját (általánosan ún. prímfaktorizációját)
meghatározhatjuk például úgy, hogy vesszük a számot, és elosztjuk a legkisebb
prímmel, ami osztója. Ezt a prímet feljegyezzük, a maradékra pedig
megismételjük a műveletet, amíg a maradék nem $1$.

Ha nem a legkisebb, hanem tetszőleges prímszámmal osztunk a folyamat során,
akkor nem a kanonikus alakot kapjuk, de a prímek növekvő sorrendbe rendezésével
megkapjuk a kanonikus alakot.

Például:
\[
\begin{array}{r|l}
  79380 & 2 \\
  39690 & 2 \\
  19845 & 3 \\
  6615 & 3 \\
  2205 & 3 \\
  735 & 3 \\
  245 & 5 \\
  49 & 7 \\
  7 & 7 \\
  1 &
\end{array}
\]

Ez alapján $79380$ kanonikus alakja $2^2 \cdot 3^4 \cdot 5 \cdot 7^2$.
\end{method2}

\textbf{Lásd még:}
